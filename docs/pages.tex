\documentclass[12pt]{article}
\usepackage[utf8]{inputenc}
\usepackage[spanish]{babel}
\usepackage{geometry}
\usepackage{hyperref}
\usepackage{enumitem}
\geometry{a4paper, margin=1in}

\title{Estructura y Organización de la Nueva Página Web de LIDSOL en Hugo Plate}
\author{LIDSOL \\ María Fernanda Ordoñez Figueroa}
\date{}

\begin{document}

\maketitle

Este documento describe mi propuesta para la estructura de la nueva página web del
Laboratorio de Investigación y Desarrollo de Software Libre (LIDSOL), utilizando la plantilla
Hugo Plate.

\section{Home}
\textbf{Objetivo:} Presentar de manera clara y atractiva la identidad de LIDSOL.
\begin{itemize}[label=--]
    \item Objetivo general del laboratorio (misión).
    \item Intereses principales del lab (software libre, desarrollo, comunidad, etc.).
    \item Imágenes de generaciones anteriores y actuales (formato carrusel o galería).
    \item Ubicación geográfica (mapa interactivo).
\end{itemize}

\section{Proyectos}
\textbf{Objetivo:} Mostrar la organización interna de LIDSOL por áreas de trabajo y los proyectos correspondientes.

Divisiones por área:
\begin{itemize}
    \item Académico/Eventos
    \item Frontend
    \item Backend
    \item Infraestructura
\end{itemize}

Cada área tiene su página con:
\begin{itemize}[label=--]
    \item Breve descripción del enfoque del área
    \item Listado de proyectos actuales, finalizados y próximos
\end{itemize}

Cada proyecto debe tener una ficha con:
\begin{itemize}[label=--]
    \item Nombre
    \item Estado (en curso, finalizado, próximo)
    \item Descripción breve
    \item Personas participantes
    \item Enlaces a repositorios o recursos
\end{itemize}

\section{Integrantes}
\textbf{Objetivo:} Presentar a los miembros del laboratorio y sus roles.  
Subsecciones:
\begin{itemize}
    \item Alumnos activos
    \item Exalumnos
    \item Profesores / Académicos
\end{itemize}

Cada perfil debe contener:
\begin{itemize}[label=--]
    \item Foto (opcional)
    \item Nombre completo
    \item Rol
    \item Descripción personal (escrita por el propio miembro)
    \item Redes de contacto (GitHub, LinkedIn, etc.) (opcional)
\end{itemize}

\section{Artículos / Posts}
\textbf{Objetivo:} Publicar contenido generado por el laboratorio.

Tipos de contenido:
\begin{itemize}[label=--]
    \item Notas técnicas
    \item Mini papers
    \item Tutoriales
    \item Opiniones
\end{itemize}

Organización sugerida:
\begin{itemize}[label=--]
    \item Uso de categorías (temas principales)
    \item Uso de etiquetas (palabras clave)
    \item Firma o autoría del miembro que publica (conexión con perfil)
\end{itemize}

\section{Media}
\textbf{Objetivo:} Compartir material didáctico y recursos desarrollados por LIDSOL.

Tipos de contenido:
\begin{itemize}[label=--]
    \item Cheat sheets
    \item Videos de cursos o charlas
    \item Presentaciones
    \item PDF de materiales educativos
\end{itemize}

Organización sugerida:
\begin{itemize}[label=--]
    \item Por tipo de recurso
    \item Por tema o proyecto asociado
\end{itemize}

\section{Eventos}
\textbf{Objetivo:} Mostrar los eventos en los que LIDSOL ha participado u organizado.

Subsecciones:
\begin{itemize}
    \item Eventos organizados por LIDSOL
    \item Eventos externos a los que hemos asistido
\end{itemize}

Cada entrada de evento debe incluir:
\begin{itemize}[label=--]
    \item Nombre del evento
    \item Fecha y ubicación
    \item Breve descripción
    \item Fotografías del evento (galería)
    \item Recursos asociados (diapositivas, videos, enlaces)
\end{itemize}

\section{Contacto}
\textbf{Objetivo:} Facilitar el contacto con el laboratorio y abrir canales de colaboración.
\begin{itemize}[label=--]
    \item Formulario de contacto
    \item Mapa de ubicación o enlace (Open Street Maps, puede ser)
    \item Correo electrónico general
    \item Redes sociales del laboratorio
    \item Invitación abierta a nuevos miembros o colaboradores externos
\end{itemize}

\section*{Resumen de Secciones}
\begin{itemize}[label=--]
    \item Home (presentación general)
    \item Proyectos (lo que hacemos)
    \item Integrantes (quiénes somos)
    \item Artículos / Posts (lo que pensamos y producimos)
    \item Media (lo que compartimos)
    \item Eventos (dónde participamos)
    \item Contacto (cómo unirse o colaborar)
\end{itemize}

\end{document}
