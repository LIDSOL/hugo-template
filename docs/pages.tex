\documentclass[12pt]{article}
\usepackage[utf8]{inputenc}
\usepackage[spanish]{babel}
\usepackage{geometry}
\usepackage{hyperref}
\usepackage{enumitem}
\geometry{a4paper, margin=1in}

\title{Estructura y Organización de la Nueva Página Web de LIDSOL en Hugo Plate}
\author{LIDSOL \\ María Fernanda Ordoñez Figueroa}
\date{}

\begin{document}

\maketitle

Este documento describe mi propuesta para la estructura de la nueva página web del
Laboratorio de Investigación y Desarrollo de Software Libre (LIDSOL), utilizando la plantilla
Hugo Plate.

\section{Home}
\textbf{Objetivo:} Presentar de manera clara y atractiva la identidad de LIDSOL.
\begin{itemize}[label=--]
    \item Objetivo general del laboratorio (misión).
    Somos un laboratorio colaborativo integrado por voluntarios, alumnos, exalumnos y académicos apasionados por el desarrollo de tecnologías libres y su impacto en la sociedad. Nuestra misión es promover la investigación y el desarrollo tecnológico a través de proyectos innovadores en diversas áreas, contribuyendo así al progreso y bienestar social.
    \item Intereses principales del lab (software libre, desarrollo, comunidad, etc.).
    En LIDSOL encontrarás un espacio para desarrollar tus habilidades y conocimientos en tecnologías libres, así como para colaborar en proyectos innovadores que buscan mejorar la sociedad.
    \begin{itemize}
        \item Desarrollo de software libre
        \item Investigación en tecnologías abiertas
        \item Apoyo a la comunidad del software libre
        \item Trabajo colaborativo
    \end{itemize}
    \item Imágenes del laboratorio y de eventos recientes (formato carrusel o galería).
    \item Ubicación geográfica (insertar video y mapa de como llegar).
\end{itemize}

\section{Acerca de}
\textbf{Objetivo:} Proporcionar información detallada sobre la historia, misión y actividades del LIDSOL.

\subsection{Sobre nosotros}
El LIDSOL es un laboratorio de la Facultad de Ingeniería de la UNAM, establecido en 2001, formado por voluntarios, alumnos, ex-alumnos y académicos interesados en el desarrollo de tecnologías libres y las discusiones sobre su impacto en la sociedad.

Buscamos promover e impulsar la investigación y desarrollo de tecnologías libres realizando proyectos afines en distintas áreas buscando el progreso y mejoramiento de la sociedad.

\subsection{Actividades}
El LIDSOL realiza múltiples actividades temporales y permanentes:
\begin{itemize}[label=--]
    \item Cursos y talleres
    \item Eventos y conferencias
    \item Instalación de software y sistemas operativos Libres
    \item Asesoría sobre el uso de licencias libres y abiertas
    \item Desarrollo de proyectos de investigación
    \item Desarrollo de tecnologías libres y abiertas
\end{itemize}

Además, el LIDSOL es el hogar del Capítulo Estudiantil de la ACM UNAM-FI.

\subsection{Organizaciones/afiliaciones}
\begin{itemize}[label=--]
    \item \textbf{Universidad Nacional Autónoma de México:} \href{http://www.unam.mx/}{UNAM}
    \item \textbf{Facultad de Ingeniería - UNAM:} \href{http://www.ingenieria.unam.mx/}{Facultad de Ingeniería}
    \item \textbf{División de Ingeniería Eléctrica - Facultad de Ingeniería - UNAM:} \href{http://www.fi-b.unam.mx/}{División de Ingeniería Eléctrica}
\end{itemize}

\subsection{Redes Sociales}
\begin{itemize}[label=--]
    \item \textbf{Correo:} \href{mailto:lidsol@protonmail.com}{lidsol@protonmail.com}
    \item \textbf{Twitter:} \href{https://twitter.com/lidsol}{@lidsol}
    \item \textbf{Facebook:} \href{https://facebook.com/lidsol.unam}{LIDSOL UNAM}
    \item \textbf{GitLab:} \href{https://gitlab.com/lidsol}{LIDSOL GitLab}
    \item \textbf{GitHub:} \href{https://github.com/lidsol}{LIDSOL GitHub}
    \item \textbf{LinkedIn:} \href{https://linkedin.com/company/lidsol/}{LIDSOL LinkedIn}
\end{itemize}

\subsection{Contacto}
\begin{itemize}[label=--]
    \item Formulario de contacto
    \item Mapa de ubicación o enlace (Open Street Maps, puede ser)
    \item Correo electrónico general
    \item Redes sociales del laboratorio
    \item Invitación abierta a nuevos miembros o colaboradores externos
\end{itemize}

\section{Proyectos}
\textbf{Nota:} En el home page hay un botón para el perfil de github del LIDSOL, mejor colocamos ese botón en la sección de Proyectos y desglosamos los proyectos dentro de esta página.

\textbf{Objetivo:} Mostrar la organización interna de LIDSOL por áreas de trabajo y los proyectos correspondientes.

Divisiones por área:
\begin{itemize}
    \item Académico/Eventos
    \item Frontend
    \item Backend
    \item Infraestructura
\end{itemize}

Cada área tiene su página con:
\begin{itemize}[label=--]
    \item Breve descripción del enfoque del área
    \item Listado de proyectos actuales, finalizados y próximos
\end{itemize}

Cada proyecto debe tener una ficha con:
\begin{itemize}[label=--]
    \item Nombre
    \item Estado (en curso, finalizado, próximo)
    \item Descripción breve
    \item Personas participantes (líderes y colaboradores)
    \item Enlaces a repositorios o recursos
\end{itemize}

\section{Integrantes}
\textbf{Objetivo:} Presentar a los miembros del laboratorio y sus roles.  
Subsecciones:
\begin{itemize}
    \item Alumnos activos
    \item Exalumnos
    \item Profesores / Académicos
\end{itemize}

Cada perfil debe contener:
\begin{itemize}[label=--]
    \item Foto (opcional)
    \item Nombre completo
    \item Rol
    \item Descripción personal (escrita por el propio miembro)
    \item Redes de contacto (GitHub, LinkedIn, etc.) (opcional)
\end{itemize}

\section{Media}
\textbf{Nota:} Esta sección está por verse, la idea es generar difusión en cuanto a investigación y material didáctico.

\textbf{Objetivo:} Compartir contenido generado por el laboratorio, otras organizaciones afines y material didáctico desarrollado por LIDSOL.

\subsection{Artículos y Publicaciones}
\begin{itemize}[label=--]
    \item Notas técnicas
    \item Mini papers
    \item Tutoriales
    \item Artículos de opinión
    \item Investigaciones y análisis
\end{itemize}

\subsection{Material Didáctico}
\begin{itemize}[label=--]
    \item Cheat sheets
    \item Guías de referencia rápida
    \item Manuales técnicos
    \item Documentación de proyectos
\end{itemize}

\subsection{Contenido Multimedia}
\begin{itemize}[label=--]
    \item Videos de cursos o charlas
    \item Presentaciones
    \item Webinars y conferencias
    \item Podcasts o grabaciones de audio
\end{itemize}

\subsection{Recursos Descargables}
\begin{itemize}[label=--]
    \item PDF de materiales educativos
    \item Plantillas y ejemplos de código
    \item Datasets y recursos de investigación
\end{itemize}

Organización sugerida:
\begin{itemize}[label=--]
    \item Uso de categorías por tipo de contenido (artículos, material didáctico, multimedia, recursos)
    \item Uso de etiquetas por temas técnicos (palabras clave)
    \item Filtros por área de especialización (Frontend, Backend, Infraestructura, Académico/Eventos)
    \item Firma o autoría del miembro que publica (conexión con perfil)
    \item Nivel de dificultad (principiante, intermedio, avanzado)
\end{itemize}

\section{Eventos}
\textbf{Objetivo:} Mostrar los eventos en los que LIDSOL ha participado u organizado.

Subsecciones:
\begin{itemize}
    \item Eventos organizados por LIDSOL
    \item Eventos externos a los que hemos asistido
\end{itemize}

Cada entrada de evento debe incluir:
\begin{itemize}[label=--]
    \item Nombre del evento
    \item Fecha y ubicación
    \item Breve descripción
    \item Fotografías del evento (galería)
    \item Recursos asociados (diapositivas, videos, enlaces)
\end{itemize}

\section*{Resumen de Secciones}
\begin{itemize}[label=--]
    \item Home (presentación general)
    \item Acerca de (historia, misión, actividades y contacto)
    \item Proyectos (lo que hacemos)
    \item Integrantes (quiénes somos)
    \item Media (lo que pensamos, producimos y compartimos)
    \item Eventos (dónde participamos)
\end{itemize}

\end{document}
